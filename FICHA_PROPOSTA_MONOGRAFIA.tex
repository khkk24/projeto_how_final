\documentclass[12pt,a4paper]{article}

% Pacotes essenciais
\usepackage[utf8]{inputenc}
\usepackage[brazilian]{babel}
\usepackage[T1]{fontenc}
\usepackage{geometry}
\usepackage{graphicx}
\usepackage{hyperref}
\usepackage{booktabs}
\usepackage{enumitem}
\usepackage{xcolor}
\usepackage{fancyhdr}
\usepackage{titlesec}

% Configurações de página
\geometry{
    a4paper,
    left=25mm,
    right=25mm,
    top=25mm,
    bottom=25mm
}

% Configuração de hyperlinks
\hypersetup{
    colorlinks=true,
    linkcolor=blue,
    filecolor=magenta,      
    urlcolor=cyan,
    pdftitle={Ficha de Proposta de Monografia},
    pdfauthor={Kokouvi Hola Kanyi, Mariano Vunge},
}

% Configuração de cabeçalho e rodapé
\pagestyle{fancy}
\fancyhf{}
\rhead{Ficha de Proposta de Monografia}
\lhead{2025}
\rfoot{\thepage}

% Espaçamento entre linhas
\renewcommand{\baselinestretch}{1.5}

% Título das seções
\titleformat{\section}
  {\normalfont\Large\bfseries\color{blue!70!black}}{\thesection}{1em}{}
\titleformat{\subsection}
  {\normalfont\large\bfseries\color{blue!50!black}}{\thesubsection}{1em}{}

\begin{document}

% Página de título
\begin{titlepage}
    \centering
    \vspace*{2cm}
    
    {\LARGE\bfseries FICHA DE PROPOSTA DE MONOGRAFIA\par}
    
    \vspace{3cm}
    
    {\Huge\bfseries Análise e Predição de Gravidade de Acidentes de Trânsito nas Rodovias Federais Brasileiras: Uma Abordagem Baseada em Machine Learning\par}
    
    \vspace{3cm}
    
    {\Large\textbf{Estudantes:}\par}
    {\large Kokouvi Hola Kanyi\par}
    {\large Mariano Vunge\par}
    
    \vspace{2cm}
    
    {\Large\textbf{Possível Orientador:}\par}
    {\LARGE\color{blue!70!black}\textbf{Prof. Anderson (ARA ANDERSON)}\par}
    
    \vfill
    
    {\large 23 de Novembro de 2025\par}
    
\end{titlepage}

\newpage
\tableofcontents
\newpage

% Identificação
\section{IDENTIFICAÇÃO}

\begin{description}[leftmargin=0pt, labelwidth=0pt, labelsep=0pt, itemsep=10pt]
    \item[\textbf{TEMA PROPOSTO:}] 
    
    {\large Análise e Predição de Gravidade de Acidentes de Trânsito nas Rodovias Federais Brasileiras: Uma Abordagem Baseada em Machine Learning}
    
    \item[\textbf{ESTUDANTES:}]
    \begin{itemize}
        \item Kokouvi Hola Kanyi
        \item Mariano Vunge
    \end{itemize}
    
    \item[\textbf{POSSÍVEL ORIENTADOR:}]
    
    {\large\textbf{Prof. Anderson (ARA ANDERSON)}}
    
    \item[\textbf{INSTITUIÇÃO:}] [Nome da Instituição]
    
    \item[\textbf{CURSO:}] [Nome do Curso]
    
    \item[\textbf{DATA:}] 23 de Novembro de 2025
\end{description}

\section{RESUMO DO TEMA}

Este trabalho propõe o desenvolvimento de um sistema de análise e predição de gravidade de acidentes de trânsito nas rodovias federais brasileiras, utilizando técnicas avançadas de Machine Learning aplicadas a dados da Polícia Rodoviária Federal (PRF) do período 2021-2025.

\subsection{Objetivo Principal}

Desenvolver modelos de classificação multiclasse capazes de prever a gravidade de acidentes (Sem Vítimas, Com Vítimas Feridas, Com Vítimas Fatais) com base em características observáveis do acidente, condições ambientais e dados geográficos.

\subsection{Justificativa}

Os acidentes de trânsito representam grave problema de saúde pública no Brasil. Um sistema preditivo pode:

\begin{itemize}
    \item Otimizar alocação de recursos de emergência
    \item Subsidiar políticas públicas de segurança viária
    \item Identificar trechos críticos para intervenções
    \item Salvar vidas através de resposta mais adequada
\end{itemize}

\section{CARACTERIZAÇÃO DA PESQUISA}

\begin{description}[leftmargin=0pt, labelwidth=0pt, labelsep=0pt, itemsep=8pt]
    \item[\textbf{Tipo:}] Pesquisa Aplicada
    \item[\textbf{Abordagem:}] Quantitativa
    \item[\textbf{Natureza:}] Descritiva e Explicativa
    \item[\textbf{Método:}] Experimental com Machine Learning
\end{description}

\section{DADOS E METODOLOGIA}

\subsection{Base de Dados}

\begin{itemize}
    \item \textbf{Fonte:} Polícia Rodoviária Federal (PRF)
    \item \textbf{Período:} 2021 - 2025
    \item \textbf{Volume:} 311.029 registros de acidentes
    \item \textbf{Mortes:} 26.039
    \item \textbf{Feridos:} 355.066
\end{itemize}

\subsection{Técnicas Aplicadas}

\begin{itemize}
    \item Análise Exploratória de Dados (EDA)
    \item Feature Engineering
    \item Classificação Multiclasse
    \item Random Forest
    \item Gradient Boosting
    \item Tratamento de Classes Desbalanceadas
\end{itemize}

\subsection{Ferramentas}

\begin{itemize}
    \item Python 3.12
    \item scikit-learn (Machine Learning)
    \item pandas (Manipulação de dados)
    \item Streamlit (Interface web)
    \item Jupyter Notebooks (Análise)
\end{itemize}

\section{RESULTADOS PRELIMINARES}

O projeto já possui implementação funcional com resultados comprovados:

\subsection{Performance do Modelo}

\begin{itemize}
    \item \textbf{Algoritmo Selecionado:} Gradient Boosting
    \item \textbf{Acurácia Geral:} 78.31\%
    \item \textbf{Features Utilizadas:} 19 (10 categóricas + 9 numéricas)
\end{itemize}

\subsection{Métricas por Classe}

\begin{table}[h]
\centering
\begin{tabular}{@{}lccc@{}}
\toprule
\textbf{Classe} & \textbf{Precision} & \textbf{Recall} & \textbf{F1-Score} \\ \midrule
Com Vítimas Feridas & 79.8\% & 96.2\% & 87.2\% \\
Sem Vítimas & 67.6\% & 25.1\% & 36.6\% \\
Com Vítimas Fatais & 47.5\% & 14.1\% & 21.8\% \\ \bottomrule
\end{tabular}
\caption{Métricas de Performance do Modelo por Classe}
\end{table}

\subsection{Top 5 Features Mais Importantes}

\begin{enumerate}
    \item Tipo do acidente (19.7\%)
    \item Quilômetro da rodovia (16.1\%)
    \item Causa do acidente (10.9\%)
    \item Hora do dia (7.1\%)
    \item Estado (UF) (6.6\%)
\end{enumerate}

\subsection{Descobertas da Análise Exploratória}

\begin{itemize}
    \item Redução de 38.4\% nos acidentes (2021-2025)
    \item Agosto é o mês mais crítico
    \item 18h é o horário com mais acidentes
    \item Sexta-feira é o dia mais perigoso
    \item Minas Gerais lidera em número de acidentes
    \item Pará tem maior taxa de mortalidade (53.86\%)
\end{itemize}

\section{CONTRIBUIÇÕES ESPERADAS}

\subsection{Científicas}

\begin{itemize}
    \item Aplicação de ML em problema real de segurança pública
    \item Metodologia de feature engineering para dados de acidentes
    \item Análise comparativa de algoritmos de classificação
    \item Estudo de técnicas para lidar com dados desbalanceados
\end{itemize}

\subsection{Práticas}

\begin{itemize}
    \item Sistema funcional para predição de gravidade
    \item Interface web interativa para visualização
    \item Insights acionáveis para políticas públicas
    \item Ferramenta para otimização de recursos de emergência
\end{itemize}

\subsection{Técnicas}

\begin{itemize}
    \item Código modular e bem documentado
    \item Pipeline reproduzível de Ciência de Dados
    \item Arquitetura escalável
    \item Boas práticas de desenvolvimento
\end{itemize}

\section{ESTRUTURA DA MONOGRAFIA}

\begin{enumerate}
    \item \textbf{Introdução} - Contextualização e objetivos
    \item \textbf{Fundamentação Teórica} - Segurança viária, ML, trabalhos relacionados
    \item \textbf{Metodologia} - Dados, preprocessamento, algoritmos, métricas
    \item \textbf{Desenvolvimento} - Arquitetura, implementação, otimização
    \item \textbf{Resultados e Discussão} - Análise exploratória, performance, aplicabilidade
    \item \textbf{Conclusão} - Síntese, contribuições, trabalhos futuros
\end{enumerate}

\section{CRONOGRAMA ESTIMADO}

\begin{table}[h]
\centering
\begin{tabular}{@{}llp{7cm}@{}}
\toprule
\textbf{Fase} & \textbf{Duração} & \textbf{Atividades} \\ \midrule
Revisão Bibliográfica & 4 semanas & Levantamento teórico, trabalhos relacionados \\
Desenvolvimento Técnico & 6 semanas & Refinamento, otimização, novos experimentos \\
Redação & 6 semanas & Escrita dos capítulos, documentação \\
Revisão e Finalização & 2 semanas & Ajustes, formatação ABNT, apresentação \\ \bottomrule
\end{tabular}
\caption{Cronograma Estimado do Projeto}
\end{table}

\textbf{Total:} 18 semanas (aproximadamente 4.5 meses)

\section{VIABILIDADE}

\subsection{Pontos Favoráveis}

\begin{itemize}
    \item[\checkmark] Dados públicos e acessíveis (PRF)
    \item[\checkmark] Implementação já funcional com resultados
    \item[\checkmark] Tecnologias open-source consolidadas
    \item[\checkmark] Escopo bem definido e realista
    \item[\checkmark] Relevância social comprovada
    \item[\checkmark] Aplicabilidade prática evidente
\end{itemize}

\subsection{Recursos Necessários}

\begin{itemize}
    \item Computador pessoal (não requer GPU)
    \item Software livre (Python e bibliotecas)
    \item Acesso à literatura acadêmica
    \item Tempo dedicado ao desenvolvimento
\end{itemize}

\section{RELEVÂNCIA DO TEMA}

\subsection{Social}

\begin{itemize}
    \item Acidentes de trânsito: 1.35 milhão de mortes/ano no mundo (OMS)
    \item Brasil: milhares de vidas perdidas anualmente
    \item Impacto econômico significativo
    \item Problema de saúde pública grave
\end{itemize}

\subsection{Acadêmica}

\begin{itemize}
    \item Integração de teoria e prática
    \item Aplicação de técnicas modernas de ML
    \item Contribuição para literatura científica
    \item Exemplo de pesquisa aplicada
\end{itemize}

\subsection{Profissional}

\begin{itemize}
    \item Desenvolvimento de habilidades em Ciência de Dados
    \item Experiência com projetos completos (end-to-end)
    \item Portfólio técnico robusto
    \item Preparação para mercado de trabalho
\end{itemize}

\section{DIFERENCIAIS DO PROJETO}

\begin{enumerate}
    \item \textbf{Abordagem Dual}: Análise exploratória + Predição (não apenas um)
    \item \textbf{Dados Reais}: 311k registros oficiais da PRF
    \item \textbf{Sistema Completo}: Análise, modelo, interface, documentação
    \item \textbf{Código Modular}: Arquitetura profissional e escalável
    \item \textbf{Resultados Comprovados}: Implementação já funcional
    \item \textbf{Aplicabilidade}: Impacto potencial em políticas públicas
\end{enumerate}

\section{SOLICITAÇÃO DE ORIENTAÇÃO}

Venho por meio deste documento solicitar formalmente a orientação do \textbf{Prof. Anderson (ARA ANDERSON)} para o desenvolvimento desta monografia.

\subsection{Justificativa da Escolha}

A escolha do Prof. Anderson como orientador se baseia em [mencionar razões específicas, como]:

\begin{itemize}
    \item Expertise em [área de especialização do professor]
    \item Experiência com projetos de Machine Learning
    \item Conhecimento em Ciência de Dados
    \item Interesse em aplicações práticas de IA
    \item [Outras razões relevantes]
\end{itemize}

\subsection{Expectativas}

Com a orientação do Prof. Anderson, espero:

\begin{itemize}
    \item Aprofundar conhecimentos teóricos em Machine Learning
    \item Refinar a metodologia científica da pesquisa
    \item Melhorar a qualidade técnica da implementação
    \item Desenvolver análise crítica aprofundada
    \item Produzir trabalho acadêmico de alto nível
\end{itemize}

\section{MATERIAIS DISPONÍVEIS}

O projeto já conta com:

\begin{itemize}
    \item[\checkmark] Repositório GitHub completo e organizado
    \item[\checkmark] Código-fonte funcional (2000+ linhas)
    \item[\checkmark] Notebooks Jupyter com análises
    \item[\checkmark] Documentação técnica detalhada
    \item[\checkmark] Interface web interativa (Streamlit)
    \item[\checkmark] Modelo treinado e salvo
    \item[\checkmark] Scripts de automação
\end{itemize}

\textbf{Repositório:} \url{https://github.com/khkk24/projeto_how_final}

\section{REFERÊNCIAS PRINCIPAIS}

\subsection{Livros}

\begin{itemize}
    \item GÉRON, A. \textit{Hands-On Machine Learning with Scikit-Learn, Keras, and TensorFlow}. O'Reilly, 2022.
    \item HASTIE, T.; TIBSHIRANI, R.; FRIEDMAN, J. \textit{The Elements of Statistical Learning}. Springer, 2009.
\end{itemize}

\subsection{Artigos}

\begin{itemize}
    \item SANTOS, K.; DIAS, J.; AMANCIO, D. A Review of Traffic Accident Prediction Models. \textit{Transportation Research Part C}, 2019.
    \item SILVA, P. B. et al. Machine Learning for Traffic Accident Severity Classification. \textit{Expert Systems with Applications}, 2022.
\end{itemize}

\subsection{Dados}

\begin{itemize}
    \item POLÍCIA RODOVIÁRIA FEDERAL. Dados Abertos - Acidentes de Trânsito. 2025.
    \item OMS. Global Status Report on Road Safety. 2023.
\end{itemize}

\section{DECLARAÇÃO}

Declaro que o tema proposto é de meu interesse acadêmico e profissional, e comprometo-me a desenvolver o trabalho com dedicação, seguindo as orientações do professor orientador e as normas acadêmicas da instituição.

Estou ciente de que a monografia deverá ser desenvolvida no período de [X] meses, com defesa prevista para [data aproximada].

\vspace{2cm}

\noindent\textbf{Data:} 23 de Novembro de 2025

\vspace{1.5cm}

\noindent\textbf{Estudantes:}

\vspace{1cm}

\noindent\rule{7cm}{0.4pt}

\noindent Kokouvi Hola Kanyi

\vspace{1cm}

\noindent\rule{7cm}{0.4pt}

\noindent Mariano Vunge

\vspace{2cm}

\noindent\textbf{Possível Orientador:}

\vspace{1cm}

\noindent\rule{7cm}{0.4pt}

\noindent Prof. Anderson (ARA ANDERSON)

\section{ANEXOS}

\begin{itemize}
    \item Anexo A: Estrutura do repositório GitHub
    \item Anexo B: Exemplos de visualizações geradas
    \item Anexo C: Prints dos notebooks com resultados
    \item Anexo D: Documentação técnica completa (ML\_README.md)
    \item Anexo E: Guia de apresentação técnica
\end{itemize}

\section{OBSERVAÇÃO IMPORTANTE}

Este projeto não parte do zero. Já existe uma implementação funcional com resultados comprovados. O período da monografia seria dedicado a:

\begin{enumerate}
    \item Aprofundamento teórico e revisão bibliográfica
    \item Refinamento técnico e experimentação de melhorias
    \item Análise crítica aprofundada dos resultados
    \item Documentação acadêmica formal
    \item Redação científica estruturada
\end{enumerate}

Isso torna o projeto altamente viável e com grande potencial de resultar em uma monografia de qualidade.

\vspace{1cm}

\section{CONTATO}

\begin{itemize}
    \item Email: [seu email]
    \item GitHub: \url{https://github.com/khkk24}
    \item Telefone: [seu telefone]
\end{itemize}

\end{document}
